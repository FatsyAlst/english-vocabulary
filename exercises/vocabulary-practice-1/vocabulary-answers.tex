\documentclass[12pt,a4paper]{article}
\usepackage[utf8]{inputenc}
\usepackage[margin=2.5cm]{geometry}
\usepackage{enumitem}
\usepackage{fancyhdr}
\usepackage{titlesec}
\usepackage{xcolor}
\usepackage{tabularx}
\usepackage{multicol}
\usepackage{array}
\usepackage[normalem]{ulem}

% Header and footer
\pagestyle{fancy}
\fancyhf{}
\setlength{\headheight}{15pt}
\lhead{English Vocabulary Practice}
\rhead{ANSWER KEY}
\cfoot{\thepage}

% Section formatting
\titleformat{\section}
{\Large\bfseries\color{green!70!black}}
{\thesection}{1em}{}

\titleformat{\subsection}
{\large\bfseries\color{green!50!black}}
{\thesubsection}{1em}{}

% Custom colors
\definecolor{dragonball}{RGB}{255, 140, 0}
\definecolor{f1red}{RGB}{200, 16, 46}
\definecolor{peaky}{RGB}{40, 40, 40}
\definecolor{correct}{RGB}{0, 128, 0}
\definecolor{noteblue}{RGB}{0, 80, 160}

\begin{document}

\begin{center}
    {\Huge\bfseries Answer Key}\\[0.5cm]
    {\Large Vocabulary Practice — Comprehensive Review}\\[0.5cm]
    \rule{\textwidth}{0.4pt}\\[0.3cm]
    {\large Practice Set 1}\\[0.2cm]
    \today
\end{center}

\vspace{0.5cm}

% ============================================================
\section{Exercise 1: Definitions — Match the Word}

\begin{enumerate}[leftmargin=2cm]
    \item A special skill or talent, often natural and hard to teach: \textbf{\color{correct}knack}
    
    \item To put something in danger; to risk losing or damaging: \textbf{\color{correct}jeopardize}
    
    \item The state of looking like or being similar to someone/something: \textbf{\color{correct}resemblance}
    
    \item Treated with excessive care; indulged or spoiled: \textbf{\color{correct}pampered}
    
    \item To remain undamaged by; to resist or endure successfully: \textbf{\color{correct}withstand}
    
    \item An enemy or opponent, especially in battle or conflict: \textbf{\color{correct}foe}
    
    \item The superlative form meaning ``most spirited'' or ``most bold'': \textbf{\color{correct}feistiest}
    
    \item Interfering in matters that don't concern you: \textbf{\color{correct}meddling}
    
    \item Extremely good; excellent or superior: \textbf{\color{correct}outstanding}
    
    \item Refusing to change opinion; determined not to give in: \textbf{\color{correct}stubborn}
    
    \item To deny or make ineffective; to nullify: \textbf{\color{correct}negate}
    
    \item A long, narrow elevated area (mountain top or raised line): \textbf{\color{correct}ridge}
\end{enumerate}

\vspace{0.5cm}
\textbf{\color{noteblue}Scoring:} 1 point each. Total: 12 points.

\newpage

% ============================================================
\section{Exercise 2: Idioms \& Expressions — Fill in the Blanks}

\begin{enumerate}[leftmargin=2cm]
    \item The manager tried to \textbf{\color{correct}sell} me \textbf{\color{correct}to} the hiring committee by highlighting all my achievements.
    
    \item After the scandal broke, the politician found himself \textbf{\color{correct}on the back foot}, constantly defending his past decisions.
    
    \item I know you're scared, but \textbf{\color{correct}search your feelings} — deep down, you know this is the right choice.
    
    \item The authorities have been accused of \textbf{\color{correct}turning a blind eye to} corruption for years.
    
    \item If you book the hotel at the last minute, you'll \textbf{\color{correct}pay through the nose} for it.
    
    \item Your business plan sounds \textbf{\color{correct}half-baked} — you haven't done enough research yet.
\end{enumerate}

\vspace{0.5cm}
\textbf{\color{noteblue}Note:} Accept grammatical variations (e.g., ``turned a blind eye to'' vs ``turning a blind eye to'').

\vspace{1cm}

% ============================================================
\section{Exercise 3: Word Forms — Complete the Table}

\begin{center}
\begin{tabularx}{\textwidth}{|l|X|X|X|X|}
\hline
\textbf{Base Word} & \textbf{Noun} & \textbf{Verb} & \textbf{Adjective} & \textbf{Adverb} \\
\hline
\textbf{feisty} & feistiness & --- & feisty, feistier, \textbf{\color{correct}feistiest} & \textbf{\color{correct}feistily} \\
\hline
\textbf{jeopardize} & \textbf{\color{correct}jeopardy} & jeopardize & --- & --- \\
\hline
\textbf{meddle} & \textbf{\color{correct}meddling}, meddler & meddle & \textbf{\color{correct}meddling} & --- \\
\hline
\textbf{negate} & \textbf{\color{correct}negation} & negate & negative & \textbf{\color{correct}negatively} \\
\hline
\textbf{stubborn} & \textbf{\color{correct}stubbornness} & --- & stubborn & \textbf{\color{correct}stubbornly} \\
\hline
\textbf{pamper} & \textbf{\color{correct}pampering} & pamper & \textbf{\color{correct}pampered} & --- \\
\hline
\textbf{withstand} & --- & withstand, \textbf{\color{correct}withstood} (past) & --- & --- \\
\hline
\textbf{steer} & \textbf{\color{correct}steering} & steer & --- & --- \\
\hline
\end{tabularx}
\end{center}

\vspace{0.5cm}
\textbf{\color{noteblue}Scoring:} 1 point per correct blank. Total: 12 points.

\newpage

% ============================================================
\section{Exercise 4: Multiple Choice — Choose the Best Answer}

\begin{enumerate}[leftmargin=2cm]
    \item The word ``outstanding'' can mean:
    
    \textbf{\color{correct}C) Both ``excellent'' and ``pending/unpaid''}
    
    \textit{Explanation: ``Outstanding'' has two main meanings: (1) exceptionally good, and (2) not yet resolved/paid.}
    
    \vspace{0.3cm}
    
    \item Which sentence uses ``foe'' correctly?
    
    \textbf{\color{correct}B) The hero faced his greatest foe in the final battle.}
    
    \textit{Explanation: ``Foe'' means enemy or opponent, typically in conflict/battle contexts.}
    
    \vspace{0.3cm}
    
    \item ``She has a knack for learning languages.'' This means she:
    
    \textbf{\color{correct}B) Has a natural talent for languages}
    
    \textit{Explanation: A ``knack'' is a natural skill or aptitude for something.}
    
    \vspace{0.3cm}
    
    \item What is the past tense of ``withstand''?
    
    \textbf{\color{correct}B) withstood}
    
    \textit{Explanation: ``Withstand'' is irregular (withstand → withstood → withstood).}
    
    \vspace{0.3cm}
    
    \item ``Turn a blind eye to'' is closest in meaning to:
    
    \textbf{\color{correct}B) Deliberately ignore something wrong}
    
    \textit{Explanation: This idiom means to consciously choose not to notice or address something.}
    
    \vspace{0.3cm}
    
    \item If a plan is ``half-baked,'' it is:
    
    \textbf{\color{correct}B) Incomplete or poorly thought out}
    
    \textit{Explanation: Like bread that's only half-cooked, a half-baked idea isn't fully developed.}
    
    \vspace{0.3cm}
    
    \item ``Parley'' is most commonly used in which context?
    
    \textbf{\color{correct}B) Negotiations between enemies}
    
    \textit{Explanation: ``Parley'' is an old-fashioned word for discussing terms with an opponent.}
    
    \vspace{0.3cm}
    
    \item A ``ridge'' can refer to all EXCEPT:
    
    \textbf{\color{correct}C) A valley between two hills}
    
    \textit{Explanation: A ridge is an elevated area; a valley is the opposite (low area between hills).}
\end{enumerate}

\vspace{0.5cm}
\textbf{\color{noteblue}Scoring:} 1 point each. Total: 8 points.

\newpage

% ============================================================
\section{Exercise 5: Sentence Completion — Use the Word in Context}

\begin{enumerate}[leftmargin=2cm]
    \item The twins bear a striking \textbf{\color{correct}resemblance} to each other — it's hard to tell them apart.
    
    \item Don't let one bad decision \textbf{\color{correct}jeopardize / negate} all the hard work you've done.
    
    \textit{(Accept either — both fit the context)}
    
    \item Despite his small size, the puppy was the \textbf{\color{correct}feistiest} of the litter, always barking at bigger dogs.
    
    \item The captain carefully \textbf{\color{correct}steered} the ship through the narrow channel.
    
    \item Her new evidence completely \textbf{\color{correct}negated} his alibi — he couldn't have been there.
    
    \item After the stressful week, she decided to \textbf{\color{correct}pamper} herself with a spa day.
    
    \item The old bridge was designed to \textbf{\color{correct}withstand} heavy traffic and harsh weather.
    
    \item Stop \textbf{\color{correct}meddling} in my business — I can make my own decisions!
    
    \item He's so \textbf{\color{correct}stubborn} that he won't change his mind even when he's clearly wrong.
    
    \item She received an award for her \textbf{\color{correct}outstanding} contribution to the field of medicine.
\end{enumerate}

\vspace{0.5cm}
\textbf{\color{noteblue}Scoring:} 1 point each. Total: 10 points.

\newpage

% ============================================================
\section{Exercise 6: Synonyms \& Antonyms}

\subsection*{Part A: Match the Synonyms}

\begin{enumerate}[leftmargin=2cm]
    \item foe → \textbf{\color{correct}D) enemy}
    \item resemblance → \textbf{\color{correct}B) similarity}
    \item jeopardize → \textbf{\color{correct}C) endanger}
    \item stubborn → \textbf{\color{correct}E) obstinate}
    \item outstanding → \textbf{\color{correct}A) excellent}
    \item knack → \textbf{\color{correct}F) talent}
\end{enumerate}

\subsection*{Part B: Write the Antonym}

\begin{enumerate}[leftmargin=2cm]
    \item foe → \textbf{\color{correct}friend / ally}
    \item pampered → \textbf{\color{correct}neglected / deprived / independent}
    \item outstanding (meaning: excellent) → \textbf{\color{correct}mediocre / average / poor}
    \item half-baked → \textbf{\color{correct}fully baked / well-planned / thorough}
    \item stubborn → \textbf{\color{correct}flexible / yielding / compliant / open-minded}
\end{enumerate}

\vspace{0.5cm}
\textbf{\color{noteblue}Scoring:} Part A: 1 point each (6 pts). Part B: 1 point each (5 pts). Total: 11 points.

\vspace{1cm}

% ============================================================
\section{Exercise 7: Error Correction}

\begin{enumerate}[leftmargin=2cm]
    \item \sout{She has a nack for playing the piano} \\
    \textit{Correction:} \textbf{\color{correct}She has a \underline{knack} for playing the piano.}\\
    \textit{(Spelling: ``knack'' not ``nack'')}
    
    \item \sout{The building withstanded the earthquake} \\
    \textit{Correction:} \textbf{\color{correct}The building \underline{withstood} the earthquake.}\\
    \textit{(Irregular verb: withstand → withstood)}
    
    \item \sout{He paid through the eye for those tickets} \\
    \textit{Correction:} \textbf{\color{correct}He paid through the \underline{nose} for those tickets.}\\
    \textit{(Correct idiom: ``pay through the nose'')}
    
    \item \sout{The company turned the blind eye to safety violations} \\
    \textit{Correction:} \textbf{\color{correct}The company turned \underline{a} blind eye to safety violations.}\\
    \textit{(Article: ``a blind eye'' not ``the blind eye'')}
    
    \item \sout{The restaurant was at the back foot} \\
    \textit{Correction:} \textbf{\color{correct}The restaurant was \underline{on} the back foot.}\\
    \textit{(Preposition: ``on'' not ``at'')}
    
    \item \sout{Your proposal sounds half-backed} \\
    \textit{Correction:} \textbf{\color{correct}Your proposal sounds half-\underline{baked}.}\\
    \textit{(Spelling: ``baked'' not ``backed'')}
    
    \item The hero defeated his greatest foe in battle. \\
    \textbf{\color{correct}No error — this sentence is correct!}
    
    \item \sout{That metaling neighbor is always asking} \\
    \textit{Correction:} \textbf{\color{correct}That \underline{meddling} neighbor is always asking.}\\
    \textit{(Spelling: ``meddling'' not ``metaling'')}
\end{enumerate}

\vspace{0.5cm}
\textbf{\color{noteblue}Scoring:} 1 point for identifying error, 1 point for correct fix. Total: 14 points.

\newpage

% ============================================================
\section{Exercise 8: Collocations — Choose the Right Partner}

\begin{enumerate}[leftmargin=2cm]
    \item bear a (\textbf{\color{correct}strong} / heavy / big) resemblance
    
    \item (\textbf{\color{correct}bitter} / sour / angry) foe
    
    \item have a (\textbf{\color{correct}knack} / knock / nack) for something
    
    \item (\textbf{\color{correct}mountain} / field / ocean) ridge
    
    \item withstand (\textbf{\color{correct}pressure} / opinion / color)
    
    \item (\textbf{\color{correct}bureaucratic} / democratic / automatic) meddling
    
    \item jeopardize someone's (\textbf{\color{correct}safety} / color / taste)
    
    \item (\textbf{\color{correct}pampered} / hammered / gathered) lifestyle
    
    \item stubborn as a (\textbf{\color{correct}mule} / mouse / monkey)
    
    \item put someone on the (\textbf{\color{correct}back} / front / side) foot
\end{enumerate}

\vspace{0.5cm}
\textbf{\color{noteblue}Scoring:} 1 point each. Total: 10 points.

\vspace{1cm}

% ============================================================
\section{Exercise 9: Translation Practice}

\begin{enumerate}[leftmargin=2cm]
    \item Ele é um garoto mimado que nunca precisou trabalhar. \\
    \textbf{\color{correct}He's a pampered kid who never had to work.}
    
    \item O prédio foi projetado para resistir a terremotos. \\
    \textbf{\color{correct}The building was designed to withstand earthquakes.}
    
    \item As autoridades fizeram vista grossa para a corrupção. \\
    \textbf{\color{correct}The authorities turned a blind eye to corruption.}
    
    \item Pagamos caro demais por ingressos de última hora. \\
    \textbf{\color{correct}We paid through the nose for last-minute tickets.}
    
    \item O gol cedo os colocou na defensiva. \\
    \textbf{\color{correct}The early goal put them on the back foot.}
    
    \item Ela tem um jeito para resolver problemas rapidamente. \\
    \textbf{\color{correct}She has a knack for solving problems quickly.}
\end{enumerate}

\vspace{0.5cm}
\textbf{\color{noteblue}Note:} Accept reasonable variations that maintain meaning and use the target vocabulary.

\newpage

% ============================================================
\section{Exercise 10: Context Clues — Guess the Word}

\begin{enumerate}[leftmargin=2cm]
    \item \textbf{Context:} The small dog kept barking at the German Shepherd...
    
    \textit{Word:} \textbf{\color{correct}feistiest} (adjective, superlative form)
    
    \vspace{0.3cm}
    
    \item \textbf{Context:} ``...he refuses to accept he might be wrong. He won't budge an inch.''
    
    \textit{Word:} \textbf{\color{correct}stubborn}
    
    \vspace{0.3cm}
    
    \item \textbf{Context:} ...the generals from both armies met to discuss terms...
    
    \textit{Word:} \textbf{\color{correct}parley} (verb, old-fashioned)
    
    \vspace{0.3cm}
    
    \item \textbf{Context:} The racing driver expertly controlled the car...
    
    \textit{Word:} \textbf{\color{correct}steer} (verb)
    
    \vspace{0.3cm}
    
    \item \textbf{Context:} ``People always say I look just like my grandmother...''
    
    \textit{Word:} \textbf{\color{correct}resemblance} (noun)
    
    \vspace{0.3cm}
    
    \item \textbf{Context:} ...the investors found it incomplete — no costs, competition, or target market.
    
    \textit{Word:} \textbf{\color{correct}half-baked} (idiom, adjective)
\end{enumerate}

\vspace{0.5cm}
\textbf{\color{noteblue}Scoring:} 1 point each. Total: 6 points.

\vspace{1cm}

% ============================================================
\section{Exercise 11: Creative Writing — Use the Words}

\textbf{\color{noteblue}Evaluation Criteria:}

\begin{itemize}
    \item Uses at least 6 vocabulary words correctly: 6 points (1 per word)
    \item Words are used in appropriate context: 3 points
    \item Grammar and coherence: 3 points
\end{itemize}

\textbf{Total possible: 12 points}

\vspace{0.5cm}

\textbf{\color{correct}Sample Answer:}

\textit{Marcus was the feistiest player on the basketball team. Despite his small size, he never backed down from any foe on the court. His stubborn determination helped him withstand pressure from bigger opponents. The coach had a knack for bringing out the best in him, even when his half-baked plays didn't work out. She never turned a blind eye to his mistakes, but she also celebrated his outstanding moments. Together, they steered the team to victory.}

\vspace{0.3cm}
\textit{(Words used: feistiest, foe, stubborn, withstand, knack, half-baked, turned a blind eye to, outstanding, steered — 9 words)}

\newpage

% ============================================================
\section{Exercise 12: Word Origins — Etymology Matching}

\begin{enumerate}[leftmargin=2cm]
    \item ridge → \textbf{\color{correct}D) From Old English ``hrycg'' (back, spine)}
    \item knack → \textbf{\color{correct}E) From an imitative word for a sharp sound/trick}
    \item parley → \textbf{\color{correct}B) From French ``parler'' (to speak)}
    \item negate → \textbf{\color{correct}C) From Latin ``negare'' (to deny)}
    \item foe → \textbf{\color{correct}A) From Old English ``fāh'' (hostile)}
\end{enumerate}

\vspace{0.5cm}
\textbf{\color{noteblue}Scoring:} 1 point each. Total: 5 points.

\vspace{1cm}

% ============================================================
\section{Exercise 13: Sentence Transformation}

\begin{enumerate}[leftmargin=2cm]
    \item The team was in a difficult, defensive position... (BACK FOOT)
    
    \textbf{\color{correct}The team was on the back foot after the early mistake.}
    
    \vspace{0.3cm}
    
    \item She has a natural talent for cooking. (KNACK)
    
    \textbf{\color{correct}She has a knack for cooking.}
    
    \vspace{0.3cm}
    
    \item The officials deliberately ignored the corruption... (BLIND EYE)
    
    \textbf{\color{correct}The officials turned a blind eye to the corruption.}
    
    \vspace{0.3cm}
    
    \item The child was spoiled and given too much attention... (PAMPERED)
    
    \textbf{\color{correct}The child was pampered by his parents.}
    
    \vspace{0.3cm}
    
    \item Don't risk your career over one bad decision. (JEOPARDIZE)
    
    \textbf{\color{correct}Don't jeopardize your career over one bad decision.}
    
    \vspace{0.3cm}
    
    \item We paid way too much for these hotel rooms. (NOSE)
    
    \textbf{\color{correct}We paid through the nose for these hotel rooms.}
\end{enumerate}

\vspace{0.5cm}
\textbf{\color{noteblue}Scoring:} 2 points each (1 for using word, 1 for correct grammar). Total: 12 points.

\newpage

% ============================================================
\section{Exercise 14: Fill the Story}

Marcus was the (1) \textbf{\color{correct}feistiest} kid in his class — small but fierce, never backing down from anyone. His (2) \textbf{\color{correct}stubborn} nature sometimes got him into trouble, but it also earned him respect.

At home, his younger brother Leo was the opposite: a (3) \textbf{\color{correct}pampered} child who had never faced real challenges. Despite their differences, they shared a striking (4) \textbf{\color{correct}resemblance} — the same bright eyes and mischievous smile.

One day, Marcus met his greatest (5) \textbf{\color{correct}foe} on the basketball court: a tall player named Derek. The game was intense, but Marcus (6) \textbf{\color{correct}withstood} the pressure and delivered an (7) \textbf{\color{correct}outstanding} performance.

Meanwhile, their (8) \textbf{\color{correct}meddling} neighbor Mrs. Chen almost (9) \textbf{\color{correct}jeopardized} the victory by calling the police about the noise! Luckily, Marcus (10) \textbf{\color{correct}steered} the situation away from disaster by charming her with his smile.

\vspace{0.5cm}
\textbf{\color{noteblue}Scoring:} 1 point each. Total: 10 points.

\vspace{2cm}

\begin{center}
    \rule{0.5\textwidth}{0.4pt}\\
    \vspace{0.3cm}
    {\Large\bfseries Total Score Summary}\\
    \vspace{0.5cm}
    \begin{tabular}{|l|c|}
    \hline
    \textbf{Exercise} & \textbf{Points} \\
    \hline
    1. Definitions & /12 \\
    2. Idioms & /6 \\
    3. Word Forms & /12 \\
    4. Multiple Choice & /8 \\
    5. Sentence Completion & /10 \\
    6. Synonyms \& Antonyms & /11 \\
    7. Error Correction & /14 \\
    8. Collocations & /10 \\
    9. Translation & /6 \\
    10. Context Clues & /6 \\
    11. Creative Writing & /12 \\
    12. Etymology & /5 \\
    13. Transformation & /12 \\
    14. Fill the Story & /10 \\
    \hline
    \textbf{TOTAL} & \textbf{/134} \\
    \hline
    \end{tabular}
    \vspace{0.5cm}
    
    \rule{0.5\textwidth}{0.4pt}\\
    \vspace{0.3cm}
    {\large Keep learning and expanding your vocabulary!}\\
    \textit{Great work on completing this comprehensive review.}
\end{center}

\end{document}

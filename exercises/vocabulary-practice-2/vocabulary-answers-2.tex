\documentclass[12pt,a4paper]{article}
\usepackage[utf8]{inputenc}
\usepackage[margin=2.5cm]{geometry}
\usepackage{enumitem}
\usepackage{fancyhdr}
\usepackage{titlesec}
\usepackage{xcolor}
\usepackage{tabularx}
\usepackage{multicol}
\usepackage{tcolorbox}

% Header and footer
\pagestyle{fancy}
\fancyhf{}
\setlength{\headheight}{15pt}
\lhead{English Vocabulary Practice}
\rhead{ANSWER KEY — SET 2}
\cfoot{\thepage}

% Section formatting
\titleformat{\section}
{\Large\bfseries\color{blue!70!black}}
{\thesection}{1em}{}

\titleformat{\subsection}
{\large\bfseries\color{blue!50!black}}
{\thesubsection}{1em}{}

% Custom colors
\definecolor{correct}{RGB}{0, 128, 0}
\definecolor{dragonball}{RGB}{255, 140, 0}
\definecolor{easygreen}{RGB}{34, 139, 34}
\definecolor{mediumyellow}{RGB}{255, 165, 0}
\definecolor{hardred}{RGB}{178, 34, 34}

\begin{document}

\begin{center}
    {\Huge\bfseries Answer Key}\\[0.5cm]
    {\Large Vocabulary Practice — Set 2}\\[0.5cm]
    \rule{\textwidth}{0.4pt}\\[0.3cm]
    {\large 12 Exercises • Complete Solutions with Explanations}\\[0.2cm]
    \today
\end{center}

\vspace{0.5cm}

% ============================================================
% EASY SECTION
% ============================================================

\begin{center}
{\Large\textcolor{easygreen}{\textbf{— EASY EXERCISES —}}}
\end{center}

\vspace{0.3cm}

\section{Exercise 1: Definitions — Match the Word}

\subsection*{Word Bank A}

\begin{enumerate}[label=\arabic*.]
    \item \textcolor{correct}{\textbf{grudge}} — A long-lasting feeling of anger/resentment
    \item \textcolor{correct}{\textbf{lurk}} — To wait somewhere secretly
    \item \textcolor{correct}{\textbf{sluggish}} — Moving slowly, lacking energy
    \item \textcolor{correct}{\textbf{defiant}} — Openly refusing to obey
    \item \textcolor{correct}{\textbf{boggling}} — Causing great surprise (mind-boggling)
    \item \textcolor{correct}{\textbf{shrewd}} — Clever at understanding/judgement
    \item \textcolor{correct}{\textbf{ambush}} — Surprise attack from hiding
    \item \textcolor{correct}{\textbf{acquainted}} — Familiar with something/someone
\end{enumerate}

\subsection*{Word Bank B}

\begin{enumerate}[label=\arabic*., start=9]
    \item \textcolor{correct}{\textbf{ponder}} — To think carefully and deeply
    \item \textcolor{correct}{\textbf{pester}} — To annoy by asking repeatedly
    \item \textcolor{correct}{\textbf{ominous}} — Suggesting something bad will happen
    \item \textcolor{correct}{\textbf{divert}} — To change direction/distract attention
    \item \textcolor{correct}{\textbf{ample}} — Enough or more than enough
    \item \textcolor{correct}{\textbf{sucker}} — Gullible person; to trick someone
    \item \textcolor{correct}{\textbf{stunt}} — Daring action; OR to prevent growth
    \item \textcolor{correct}{\textbf{outrageous}} — Shocking/offensive; OR excessively bold
\end{enumerate}

\vspace{0.5cm}

% ============================================================
\section{Exercise 2: Synonyms \& Antonyms}

\subsection*{Part A: Synonyms}

\begin{enumerate}[label=\arabic*.]
    \item shrewd — \textcolor{correct}{\textbf{B) astute}} (both mean clever at judgement)
    \item sluggish — \textcolor{correct}{\textbf{C) lethargic}} (both mean lacking energy)
    \item ample — \textcolor{correct}{\textbf{C) abundant}} (both mean plentiful)
    \item defiant — \textcolor{correct}{\textbf{B) rebellious}} (both mean refusing to obey)
    \item ponder — \textcolor{correct}{\textbf{B) contemplate}} (both mean to think deeply)
    \item ominous — \textcolor{correct}{\textbf{C) threatening}} (both suggest danger/bad outcomes)
    \item lurk — \textcolor{correct}{\textbf{B) skulk}} (both mean to wait/move secretively)
    \item outrageous — \textcolor{correct}{\textbf{C) scandalous}} (both mean shocking)
\end{enumerate}

\subsection*{Part B: Antonyms}

\begin{enumerate}[label=\arabic*., start=9]
    \item sluggish — \textcolor{correct}{\textbf{B) brisk}} (slow vs. quick/energetic)
    \item ample — \textcolor{correct}{\textbf{C) scarce}} (plentiful vs. lacking)
    \item defiant — \textcolor{correct}{\textbf{C) compliant}} (rebellious vs. obedient)
    \item shrewd — \textcolor{correct}{\textbf{C) gullible}} (clever vs. easily fooled)
    \item ominous — \textcolor{correct}{\textbf{C) auspicious}} (threatening vs. promising good)
    \item acquainted — \textcolor{correct}{\textbf{C) unfamiliar}} (knowing vs. not knowing)
\end{enumerate}

\vspace{0.5cm}

% ============================================================
\section{Exercise 3: Word Forms — Complete the Table}

\begin{enumerate}[label=(\arabic*)]
    \item \textcolor{correct}{\textbf{pondering / ponderation}} (noun forms)
    \item \textcolor{correct}{\textbf{divert}}
    \item \textcolor{correct}{\textbf{defiant}}
    \item \textcolor{correct}{\textbf{shrewdly}}
    \item \textcolor{correct}{\textbf{sluggish}}
    \item \textcolor{correct}{\textbf{acquaintance}}
    \item \textcolor{correct}{\textbf{ambush}}
    \item \textcolor{correct}{\textbf{grudging}}
    \item \textcolor{correct}{\textbf{outrageous}}
    \item \textcolor{correct}{\textbf{pest / pestering}} (noun forms)
\end{enumerate}

\newpage

% ============================================================
% MEDIUM SECTION
% ============================================================

\begin{center}
{\Large\textcolor{mediumyellow}{\textbf{— MEDIUM EXERCISES —}}}
\end{center}

\vspace{0.3cm}

\section{Exercise 4: Fill in the Blanks — Context Sentences}

\begin{enumerate}[label=\arabic*.]
    \item The mind-\textcolor{correct}{\textbf{boggling}} complexity...
    \item ...I felt heavy and \textcolor{correct}{\textbf{sluggish}}...
    \item A crocodile was \textcolor{correct}{\textbf{lurking}}...
    \item ...set up an \textcolor{correct}{\textbf{ambush}}...
    \item As a \textcolor{correct}{\textbf{shrewd}} businesswoman...
    \item ...remained \textcolor{correct}{\textbf{defiant}} and refused...
    \item She's held a \textcolor{correct}{\textbf{grudge}}...
    \item ...become fully \textcolor{correct}{\textbf{acquainted}}...
    \item ...time to \textcolor{correct}{\textbf{ponder}} over...
    \item Please stop \textcolor{correct}{\textbf{pestering}} me...
    \item ...an \textcolor{correct}{\textbf{ominous}} atmosphere...
    \item ...tried to \textcolor{correct}{\textbf{divert}} public attention...
    \item There is \textcolor{correct}{\textbf{ample}} parking space...
    \item The con artist \textcolor{correct}{\textbf{suckered}} him...
    \item ...can \textcolor{correct}{\textbf{stunt}} a person's physical development.
    \item ...prices are absolutely \textcolor{correct}{\textbf{outrageous}}!
\end{enumerate}

\vspace{0.5cm}

% ============================================================
\section{Exercise 5: Collocations — Choose the Best Match}

\begin{enumerate}[label=\arabic*.]
    \item mind-\textcolor{correct}{\textbf{B) boggling}}
    \item bear/hold/harbour a \textcolor{correct}{\textbf{A) grudge}}
    \item \textcolor{correct}{\textbf{C) lurk}} in the shadows
    \item \textcolor{correct}{\textbf{B) divert}} somebody's attention
    \item \textcolor{correct}{\textbf{C) sluggish}} economy/growth/demand
    \item \textcolor{correct}{\textbf{C) shrewd}} businessman/judge
    \item \textcolor{correct}{\textbf{A) ample}} time/space/parking
    \item \textcolor{correct}{\textbf{C) outrageous}} prices/behaviour
    \item \textcolor{correct}{\textbf{B) ominous}} warning/sign/clouds
    \item death-defying \textcolor{correct}{\textbf{C) stunt}}
\end{enumerate}

\vspace{0.5cm}

% ============================================================
\section{Exercise 6: Dragon Ball Z Scene Comprehension}

\begin{enumerate}[label=\arabic*.]
    \item \textcolor{correct}{\textbf{suckers}} — Vegeta calls them easily fooled people

    \item \textcolor{correct}{\textbf{lurking}} — Frieza's men have been lurking around the villages

    \item He says he's not \textcolor{correct}{\textbf{sluggish}} like them — he's been alert and watching

    \item \textcolor{correct}{\textbf{Outrageous}} — Frieza's demands for mercy are excessive/shocking

    \item \textcolor{correct}{\textbf{Divert}} their attention — make them think they're going east while circling back

    \item \textcolor{correct}{\textbf{ponder}} — Vegeta says he's had ample time to ponder this strategy

    \item \textcolor{correct}{\textbf{defiant}} — describes his body language when he refuses to run

    \item \textcolor{correct}{\textbf{shrewd}} — Gohan calls the plan shrewd, showing good judgement
\end{enumerate}

\newpage

% ============================================================
\section{Exercise 7: Register \& Formality}

\subsection*{Part A: Formal vs. Informal Contexts}

\begin{enumerate}[label=\arabic*.]
    \item \textcolor{correct}{\textbf{B) ponder}}

    \textit{Why: "Ponder" is more formal and literary, appropriate for business presentations. "Think about" is too casual, "check out" is slang.}

    \item \textcolor{correct}{\textbf{B) suckered}}

    \textit{Why: "Suckered" is informal slang, perfect for casual text messages. "Deceived" and "beguiled" are too formal for this context.}

    \item \textcolor{correct}{\textbf{C) sluggish}}

    \textit{Why: "Sluggish" is the standard economics term for slow growth. "Slow" is too basic, "lazy" is informal and personifies the economy inappropriately.}

    \item \textcolor{correct}{\textbf{C) lurked}}

    \textit{Why: "Lurked" has sinister connotations perfect for fantasy writing. "Waited" and "stayed" are too neutral and lack atmosphere.}
\end{enumerate}

\subsection*{Part B: Word Pairs — When to Use Each}

\begin{enumerate}[label=\arabic*., start=5]
    \item \textbf{ponder} vs. \textbf{think about}

    \textcolor{correct}{"Ponder" is more formal/literary, implying deeper, prolonged thought. "Think about" is everyday usage, neutral register. Use "ponder" in writing, formal speech; "think about" in casual conversation.}

    \item \textbf{ample} vs. \textbf{enough}

    \textcolor{correct}{"Ample" is more formal and emphasizes generosity/abundance (more than sufficient). "Enough" is neutral and just means sufficient. "Ample parking" sounds professional; "enough parking" is conversational.}

    \item \textbf{defiant} vs. \textbf{stubborn}

    \textcolor{correct}{"Defiant" implies active resistance against authority—often heroic or dramatic. "Stubborn" is more neutral/negative, suggesting unreasonable inflexibility. A rebel is defiant; a child refusing vegetables is stubborn.}

    \item \textbf{ominous} vs. \textbf{scary}

    \textcolor{correct}{"Ominous" is literary, suggests foreboding about future events—atmospheric and sophisticated. "Scary" is casual, immediate, and often childlike. "Ominous clouds" is literary; "scary movie" is everyday speech.}
\end{enumerate}

\vspace{0.5cm}

% ============================================================
\section{Exercise 8: Listening Comprehension Prompts}

\begin{tcolorbox}[colback=blue!5!white, colframe=blue!50!black, title={\textbf{Reader's Script — Read These Aloud}}]
\textit{If you are the reader/teacher, use this page to read the sentences aloud. The listener should write down the vocabulary word they hear.}
\end{tcolorbox}

\subsection*{Part A: Single Word Identification — Full Sentences to Read}

\begin{enumerate}[label=\arabic*.]
    \item "The complexity of the human brain is absolutely mind-\textbf{boggling}."

    \textcolor{correct}{Answer: \textbf{boggling}}

    \item "Someone was \textbf{lurking} in the alley behind the restaurant."

    \textcolor{correct}{Answer: \textbf{lurking}}

    \item "She's been holding a \textbf{grudge} against him since high school."

    \textcolor{correct}{Answer: \textbf{grudge}}

    \item "The economy has been quite \textbf{sluggish} this quarter."

    \textcolor{correct}{Answer: \textbf{sluggish}}

    \item "He made a \textbf{shrewd} investment decision that paid off."

    \textcolor{correct}{Answer: \textbf{shrewd}}

    \item "There's \textbf{ample} evidence to support the theory."

    \textcolor{correct}{Answer: \textbf{ample}}

    \item "The sky looked \textbf{ominous} before the hurricane hit."

    \textcolor{correct}{Answer: \textbf{ominous}}

    \item "She remained \textbf{defiant} despite the pressure to conform."

    \textcolor{correct}{Answer: \textbf{defiant}}
\end{enumerate}

\subsection*{Part B: Dictation Practice — Full Sentences to Read}

\textit{Read each sentence twice at normal speed. The listener writes the entire sentence.}

\begin{enumerate}[label=\arabic*., start=9]
    \item \textbf{Read:} "The guerrillas ambushed the convoy near the mountain pass."

    \textcolor{correct}{Key word: \textbf{ambushed} — past tense, note the -ed ending}

    \item \textbf{Read:} "I need more time to ponder this difficult decision."

    \textcolor{correct}{Key word: \textbf{ponder} — base form after "to"}

    \item \textbf{Read:} "Don't let them sucker you into buying unnecessary insurance."

    \textcolor{correct}{Key word: \textbf{sucker} — informal verb, note double consonant}

    \item \textbf{Read:} "Are you acquainted with the new safety regulations?"

    \textcolor{correct}{Key word: \textbf{acquainted} — note the "acq-" beginning}
\end{enumerate}

\newpage

% ============================================================
% HARD SECTION
% ============================================================

\begin{center}
{\Large\textcolor{hardred}{\textbf{— HARD EXERCISES —}}}
\end{center}

\vspace{0.3cm}

\section{Exercise 9: Trap Sentences — Spot the Misuse}

\begin{enumerate}[label=\arabic*.]
    \item "I need to \textbf{ponder} quickly about what to order for lunch."

    \textcolor{correct}{\textbf{Error:} "ponder quickly" is contradictory}

    \textcolor{correct}{\textbf{Why:} "Ponder" implies slow, deep, prolonged thought. You cannot ponder "quickly" — that defeats the word's meaning.}

    \textcolor{correct}{\textbf{Better:} "decide quickly" or "think quickly"}

    \vspace{0.2cm}

    \item "The cheerful music created an \textbf{ominous} atmosphere at the birthday party."

    \textcolor{correct}{\textbf{Error:} Semantic contradiction}

    \textcolor{correct}{\textbf{Why:} "Ominous" means threatening/foreboding. Cheerful music creates the opposite effect. The sentence contradicts itself.}

    \textcolor{correct}{\textbf{Better:} "festive" or "joyful" atmosphere}

    \vspace{0.2cm}

    \item "After drinking three cups of coffee, I felt very \textbf{sluggish} and energetic."

    \textcolor{correct}{\textbf{Error:} "sluggish and energetic" are opposites}

    \textcolor{correct}{\textbf{Why:} You can't be sluggish AND energetic. Coffee typically makes you energetic, not sluggish.}

    \textcolor{correct}{\textbf{Better:} "alert and energetic" OR remove "sluggish" entirely}

    \vspace{0.2cm}

    \item "She \textbf{lurked} openly in the middle of the crowded plaza."

    \textcolor{correct}{\textbf{Error:} "lurked openly" is contradictory}

    \textcolor{correct}{\textbf{Why:} "Lurk" means to hide secretly, wait in concealment. You cannot lurk "openly" — that's impossible by definition.}

    \textcolor{correct}{\textbf{Better:} "stood" or "waited" openly}

    \vspace{0.2cm}

    \item "He made a \textbf{shrewd} decision by investing all his money in a random cryptocurrency without research."

    \textcolor{correct}{\textbf{Error:} Action contradicts "shrewd"}

    \textcolor{correct}{\textbf{Why:} "Shrewd" means showing good judgement. Investing randomly without research is the opposite of shrewd — it's foolish.}

    \textcolor{correct}{\textbf{Better:} "reckless" or "foolish" decision}

    \vspace{0.2cm}

    \item "The obedient student was very \textbf{defiant} and always followed the rules."

    \textcolor{correct}{\textbf{Error:} "defiant" contradicts "obedient" and "followed the rules"}

    \textcolor{correct}{\textbf{Why:} "Defiant" means refusing to obey. An obedient, rule-following student is the opposite of defiant.}

    \textcolor{correct}{\textbf{Better:} "compliant" or "cooperative"}

    \vspace{0.2cm}

    \item "We had \textbf{ample} time—only 30 seconds to complete the entire exam."

    \textcolor{correct}{\textbf{Error:} 30 seconds is not "ample"}

    \textcolor{correct}{\textbf{Why:} "Ample" means more than enough. 30 seconds for an entire exam is extremely insufficient, not ample.}

    \textcolor{correct}{\textbf{Better:} "barely any time" or "insufficient time"}

    \vspace{0.2cm}

    \item "The soldiers launched a surprise \textbf{ambush} by marching openly down the main road in broad daylight."

    \textcolor{correct}{\textbf{Error:} Open marching contradicts "ambush"}

    \textcolor{correct}{\textbf{Why:} An "ambush" requires hiding and surprise. Marching openly in daylight eliminates the surprise element entirely.}

    \textcolor{correct}{\textbf{Better:} "attack" or "assault" — or describe hiding before the ambush}
\end{enumerate}

\newpage

% ============================================================
\section{Exercise 10: Error Correction}

\begin{enumerate}[label=\arabic*.]
    \item \textbf{sluggishly} → \textcolor{correct}{\textbf{sluggish}}

    \textit{Explanation: After "felt" (linking verb), use an adjective, not an adverb.}

    \vspace{0.2cm}

    \item \textbf{lurked} → \textcolor{correct}{\textbf{lurking}}

    \textit{Explanation: Present perfect continuous requires -ing form: "has been lurking"}

    \vspace{0.2cm}

    \item \textbf{defiantly} → \textcolor{correct}{\textbf{defiant}}

    \textit{Explanation: "Look" is a noun here, requiring an adjective modifier: "a defiant look"}

    \vspace{0.2cm}

    \item \textbf{grudging} → \textcolor{correct}{\textbf{grudge}}

    \textit{Explanation: The expression is "hold a grudge" (noun), not "hold a grudging" (adjective)}

    \vspace{0.2cm}

    \item \textbf{acquaint} → \textcolor{correct}{\textbf{acquainted}}

    \textit{Explanation: "Become acquainted" requires the past participle adjective form}

    \vspace{0.2cm}

    \item \textbf{amply} → \textcolor{correct}{\textbf{ample}}

    \textit{Explanation: "Time" is a noun, requiring an adjective modifier: "ample time"}

    \vspace{0.2cm}

    \item \textbf{ambushed on} → \textcolor{correct}{\textbf{ambushed}}

    \textit{Explanation: "Ambush" is transitive; no preposition needed: "ambushed the convoy"}

    \vspace{0.2cm}

    \item \textbf{pondering} → \textcolor{correct}{\textbf{ponder}}

    \textit{Explanation: After "to" (infinitive marker), use base form: "to ponder"}
\end{enumerate}

\vspace{0.5cm}

% ============================================================
\section{Exercise 11: Translation \& Production}

\subsection*{Part A: English to Portuguese (Model Answers)}

\begin{enumerate}[label=\arabic*.]
    \item \textcolor{correct}{A complexidade \textbf{alucinante/espantosa} do universo continua a surpreender os cientistas.}

    \item \textcolor{correct}{Ela \textbf{guardou rancor/mágoa} contra ele por anos após a discussão deles.}

    \item \textcolor{correct}{As nuvens \textbf{ameaçadoras/sinistras} sugeriam que uma tempestade terrível estava se aproximando.}

    \item \textcolor{correct}{Ele era \textbf{astuto/sagaz} demais para cair na lábia enganosa de vendas deles.}

    \item \textcolor{correct}{O governo tentou \textbf{desviar} a atenção pública do escândalo.}
\end{enumerate}

\subsection*{Part B: Portuguese to English (Model Answers)}

\begin{enumerate}[label=\arabic*., start=6]
    \item \textcolor{correct}{He felt very \textbf{sluggish} and heavy after the meal.}

    \item \textcolor{correct}{The guerrillas \textbf{ambushed} them near the bridge.}

    \item \textcolor{correct}{I'm not personally \textbf{acquainted} with her.}

    \item \textcolor{correct}{There is \textbf{ample} free parking available.}

    \item \textcolor{correct}{Someone was \textbf{lurking} in the shadows.}
\end{enumerate}

\subsection*{Part C: Sentence Creation (Model Answers)}

\begin{enumerate}[label=\arabic*., start=11]
    \item \textbf{pester}: \textcolor{correct}{My little brother kept pestering me to play video games with him until I finally gave in.}

    \item \textbf{defiant}: \textcolor{correct}{Despite the teacher's warnings, the defiant student continued to use her phone in class.}

    \item \textbf{sucker (verb)}: \textcolor{correct}{The scammers suckered elderly people into giving them their bank details.}

    \item \textbf{stunt (noun)}: \textcolor{correct}{The motorcycle stunt in the James Bond film required months of practice.}

    \item \textbf{outrageous}: \textcolor{correct}{The restaurant charged an outrageous \$50 for a simple salad.}
\end{enumerate}

\newpage

% ============================================================
\section{Exercise 12: Advanced Reading \& Analysis}

\begin{enumerate}[label=\arabic*.]
    \item \textbf{Irony with "shrewd":}

    \textcolor{correct}{The irony is that Marcus \textit{considers himself} shrewd, but the story immediately reveals he was "suckered" — the opposite of shrewd behaviour. The author sets up the reader's expectation with "shrewd" then subverts it, highlighting Marcus's overconfidence and the gap between self-perception and reality.}

    \vspace{0.5cm}

    \item \textbf{Connotation of "lurking":}

    \textcolor{correct}{"Lurking" has predatory, sinister connotations — it suggests something hiding with malicious intent, like a predator waiting for prey. This foreshadows that the con artists will "attack" (scam) victims. The word choice immediately signals danger and sets an ominous tone.}

    \vspace{0.5cm}

    \item \textbf{Marcus's emotional/physical state:}

    \textcolor{correct}{Two words: \textbf{"sluggish"} and \textbf{"ominous"} (thoughts). "Sluggish" shows his physical depression — lacking energy, unable to move forward. "Ominous thoughts" suggests dark, threatening mental states. Together they reinforce total loss — both physical vitality and mental wellbeing have been destroyed.}

    \vspace{0.5cm}

    \item \textbf{Family reactions comparison:}

    \textcolor{correct}{
    \begin{itemize}
        \item Marcus: \textbf{sluggish}, uncertain, \textbf{acquainted} with failure — passive, defeated
        \item Wife: holds a \textbf{grudge} — focused on resentment toward the con artists
        \item Daughter: \textbf{defiant} — actively resistant, optimistic, refusing to accept defeat
    \end{itemize}
    The vocabulary shows three different coping mechanisms: depression, anger, and resilience.}

    \vspace{0.5cm}

    \item \textbf{"Outrageous" — positive or negative:}

    \textcolor{correct}{Negative in context. While "outrageous" can sometimes mean impressively excessive (like "outrageous fashion"), here it means \textit{unbelievably unrealistic/suspicious}. The 300\% returns were so outrageously high that they should have been a red flag. The word emphasizes that the offer was too good to be true — anyone less greedy would have recognized the warning sign.}
\end{enumerate}

\vspace{1cm}

\begin{center}
\rule{0.8\textwidth}{0.4pt}\\[0.3cm]
{\Large\textbf{End of Answer Key}}\\[0.2cm]
\textit{Score Guide: EASY (1-3): 0-30 points | MEDIUM (4-8): 31-70 points | HARD (9-12): 71-100 points}
\end{center}

\end{document}
